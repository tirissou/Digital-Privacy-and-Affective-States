%%
%% This is file `sample-acmtog.tex',
%% generated with the docstrip utility.
%%
%% The original source files were:
%%
%% samples.dtx  (with options: `acmtog')
%%
%% IMPORTANT NOTICE:
%%
%% For the copyright see the source file.
%%
%% Any modified versions of this file must be renamed
%% with new filenames distinct from sample-acmtog.tex.
%%
%% For distribution of the original source see the terms
%% for copying and modification in the file samples.dtx.
%%
%% This generated file may be distributed as long as the
%% original source files, as listed above, are part of the
%% same distribution. (The sources need not necessarily be
%% in the same archive or directory.)
%%
%% The first command in your LaTeX source must be the \documentclass command.
\documentclass[acmtog]{acmart}

%%
%% \BibTeX command to typeset BibTeX logo in the docs
\AtBeginDocument{%
  \providecommand\BibTeX{{%
    \normalfont B\kern-0.5em{\scshape i\kern-0.25em b}\kern-0.8em\TeX}}}

%% Rights management information.  This infrmation is sent to you
%% when you complete the rights form.  These commands have SAMPLE
%% values in them; it is your responsibility as an author to replace
%% the commands and values with those provided to you when you
%% complete the rights form.
\setcopyright{acmcopyright}
\copyrightyear{2021}
\acmYear{2021}
%%\acmDOI{10.1145/1122445.1122456}


%%
%% These commands are for a JOURNAL article.
\acmJournal{TOG}
\acmVolume{69}
\acmNumber{420}
\acmArticle{314159}
\acmMonth{5}

%%
%% Submission ID.
%% Use this when submitting an article to a sponsored event. You'll
%% receive a unique submission ID from the organizers
%% of the event, and this ID should be used as the parameter to this command.
%%\acmSubmissionID{123-A56-BU3}

%%
%% The majority of ACM publications use numbered citations and
%% references.  The command \citestyle{authoryear} switches to the
%% "author year" style.
%%
%% If you are preparing content for an event
%% sponsored by ACM SIGGRAPH, you must use the "author year" style of
%% citations and references.
\citestyle{acmauthoryear}

%%
%% end of the preamble, start of the body of the document source.
\begin{document}

%%
%% The "title" command has an optional parameter,
%% allowing the author to define a "short title" to be used in page headers.
\title{The Importance of Affective States in Privacy Calculus}

%%
%% The "author" command and its associated commands are used to define
%% the authors and their affiliations.
%% Of note is the shared affiliation of the first two authors, and the
%% "authornote" and "authornotemark" commands
%% used to denote shared contribution to the research.
\author{Thibault Irissou}
\affiliation{%
  \institution{Oberlin College}
  \city{Oberlin}
  \country{USA}}
\email{tirissou@oberlin.edu}

\author{Tara Bobinac}
\affiliation{%
  \institution{Oberlin College}
  \city{Oberlin}
  \country{USA}}
\email{tbobinac@oberlin.edu}

\author{Julia Cornell}
\affiliation{%
  \institution{Oberlin College}
  \city{Oberlin}
  \country{USA}}
\email{jcornel2@oberlin.edu}

\author{Riley Chinburg}
\affiliation{%
  \institution{Oberlin College}
  \city{Oberlin}
  \country{USA}}
\email{rchinbur@oberlin.edu}

\graphicspath{ {../images/} }


%%
%% By default, the full list of authors will be used in the page
%% headers. Often, this list is too long, and will overlap
%% other information printed in the page headers. This command allows
%% the author to define a more concise list
%% of authors' names for this purpose.
\renewcommand{\shortauthors}{Irissou, Bobinac, Chinburg, and Cornell}

%%
%% The abstract is a short summary of the work to be presented in the
%% article.
\begin{abstract}
    Research historically understates the importance of affective states within privacy calculus frameworks. However, positive and negative affective states are critical in shaping human behavior. We aim to explore whether positive or negative affect more strongly reinforces privacy conscious behavior.
\end{abstract}

%%
%% The code below is generated by the tool at http://dl.acm.org/ccs.cfm.
%% Please copy and paste the code instead of the example below.
%%
%%\begin{CCSXML}
%%<ccs2012>
%% <concept>
%%  <concept_id>10010520.10010553.10010562</concept_id>
%%  <concept_desc>Computer systems organization~Embedded systems</concept_desc>
%%  <concept_significance>500</concept_significance>
%% </concept>
%% <concept>
%%  <concept_id>10010520.10010575.10010755</concept_id>
%%  <concept_desc>Computer systems organization~Redundancy</concept_desc>
%%  <concept_significance>300</concept_significance>
%% </concept>
%% <concept>
%%  <concept_id>10010520.10010553.10010554</concept_id>
%%  <concept_desc>Computer systems organization~Robotics</concept_desc>
%%  <concept_significance>100</concept_significance>
%% </concept>
%% <concept>
%%  <concept_id>10003033.10003083.10003095</concept_id>
%%  <concept_desc>Networks~Network reliability</concept_desc>
%%  <concept_significance>100</concept_significance>
%% </concept>
%%</ccs2012>
%%\end{CCSXML}
%%
%%\ccsdesc[500]{Computer systems organization~Embedded systems}
%%\ccsdesc[300]{Computer systems organization~Redundancy}
%%\ccsdesc{Computer systems organization~Robotics}
%%\ccsdesc[100]{Networks~Network reliability}
%%

%%
%% Keywords. The author(s) should pick words that accurately describe
%% the work being presented. Separate the keywords with commas.
%%\keywords{datasets, neural networks, gaze detection, text tagging}


%%
%% This command processes the author and affiliation and title
%% information and builds the first part of the formatted document.
\maketitle

\section{Introduction}

Privacy literacy does not necessarily translate into privacy conscious behavior
— the great privacy paradox. To describe this seemingly irrational
decision-making process, many studies have explored different models of this
privacy calculus \cite{Kehr2015} \cite{Jian2013}. These privacy-infringing
decisions despite purported privacy concerns may result in negative affective
states such as guilt and shame. \\

Question 1: To what extent are guilt or shame the result of privacy infringing
decision-making? Is there a correlation between the privacy literacy of an
individual and the amount of guilt/shame?

Hypothesis: We believe that guilt and shame will be felt in these situations. We
hypothesize that the privacy literacy of an individual will positively correlate
with the extent of these negative affective states. \\

Question 2: Does engaging in privacy-preserving behavior result in positive
affective states?

Hypothesis 2: We predict that such affectives states will not correlate that
strongly with privacy preserving behavior. \\

Question 3: Do these affectives states motivate people to adopt more
privacy-preserving behavior?

Hypothesis 3: We predict that both negative affective states are not effective
and that positive affective states are effective at motivating such behavior.


\section{Related Works}
\subsection{Privacy paradox, privacy calculus and affects}
The privacy paradox’s existence is for certain, as Barth and de Jong establish
in their 2017 literature review. Contextual and environmental circumstances
impact the rationality of decision-making, especially online. Studies
demonstrate that irrationality in decisions-making regarding privacy is not
necessarily circumvented through privacy literacy. \cite{Barth2017}

Some research took the direction of looking at privacy-preserving behavior as
being the irrational decision [Cohen]. However, much recent literature assumes
that decisions being made are rational insomuch as there is a level of privacy
calculus that individuals perform when assessing the risks and rewards of
disclosing information or violating their privacy
\cite{Kehr2015}\cite{Jian2013}. Affect and affective states (emotions and moods)
have been included in recent models of privacy calculus. These frameworks
provide the basis for how we conceptualize affective impact on decision making
regarding privacy.

\subsection{Negative affective states}
Teroni and Deonna explore in their paper the differentiation of shame and guilt.
Their distinctions are the foundation for how we conceptualize negative
affective states. We use the questionnaire as detailed in \cite{Rusch2007} as a
model for how we build our survey.
\subsection{Positive affective states and Privacy}
Research places importance on the perceived social benefits that come from
self-disclosure of information online in synchronous interactive
environments\cite{Jian2013}. Such perception of benefits generates positive
affective states from such divulsion of information.

In line with the investigation of positive affective states as a motivating
factor in behavior shaping \cite{Taneja2014}, we are interested in how positive
affective states as a result to making privacy-conscious and privacy-preserving
decisions correlate with frequency of such decision-making.


\section{Methodology}
This is a mixed methods study. Surveys are sent out first to find correlations
between privacy preserving behavior and the affective states that reinforce them
to varying success. Interviews will be conducted with respondents from the
survey who want to participate.
\subsection{Questionnaires}
\nopagebreak The survey is designed to answer:
\begin{itemize}
    \item What privacy protecting behavior respondents actively engage in (from
	a list of common ones),
    \item How regular and consistent this behavior is
    \item The extent to which respondents feel good or bad when they engage in
	privacy-protective behavior
    \item The extent to which respondents feel good or bad when they do not
	engage in privacy-protective behavior
\end{itemize}

\noindent The survey was modeled after multiple established surveys:
\begin{itemize}
    \item Surveys to gauge privacy perception and behavior
    \item Use established questionnaires for measuring negative affects
	\cite{Rusch2007}
    \item Design questions with best practices for digital methodology in mind
	\cite{Hampton2017}
\end{itemize}

Quantitative data is analyzed to find correlations between behavior and affect.
\subsection{Interviews}
Findings in the quantitative data will be used by interviewers to help guide
interviewees through semi-structured interviews. The objective of the interviews
is to \textbf{a)} collect detailed descriptions of the affective states that are
experienced before, during, and after engaging in privacy-related behavior, and
\textbf{b)} explore the extent to which privacy-related behavior is shaped by
the affective states surrounding them. Prior to addressing these, the
interviewer will establish what the interviewee's privacy-related habits are and
the extent of their knowledge about privacy.

The main guiding questions for the interviews are as follow:\\

\textbf{QUESTION 1}: How private and confidential is your information when
you’re online? Are there certain activities that are riskier/safer than others?

\textbf{QUESTION 2}: Do you take any protective measures when using the
internet? Why or why not?

\textbf{QUESTION 3}: How does the way that you currently browse the internet
make you feel in regards to how secure you believe your information to be?

\section{Findings}
\subsection{Survey}
67 responses were collected from people over the age of 18. Respondents were
primarily  college students, of which a majority are enrolled at Oberlin
College.

\subsection{Respondents' Privacy Concerns}
On average, respondents reported two privacy concerns of the five that were
suggested. The most common concern---password security---was three times as
popular than the least common concern---email tracking.

Fig. \ref{Univariate Distribution Respondent Concerns} presents the frequency
distribution of all five possible concerns.

\begin{figure}[H]
    \includegraphics[width=0.5\textwidth]{concerns}
    \caption{Frequency distribution for respondents' self-reported privacy
    concerns}
    \label{Univariate Distribution Respondent Concerns}
\end{figure}

\subsection{Ad Tracking}
A strong relationship exists such that \emph{the better an individual feels when
they delete cookies, the more frequently they delete them}. This was determined
by obtaining the gamma correlation value of magnitude .821 (p<.01,**) between
ordinal values for ``How often do you delete cookies on the device you most
regularly use for browsing?'' and ``How do you feel when you delete cookies?''.
Despite the strength of the relationship, the Chi-Square p-value of .178
indicates that the null hypothesis cannot be ruled out.

\subsection{Privacy and Social Media}

\begin{figure}[H]
    \includegraphics[width=0.5\textwidth]{univariate social media practices}
    \caption{Respondents' actual privacy practices regarding on social media.
	Only respondents who indicated concern about social media privacy were
	asked to answer this question.}
    \label{Univariate Distribution Respondent Concerns}
\end{figure}

A strong relationship exists between \emph{the proportion of social media for
which an individual has set up privacy settings and how they feel when they are
setting up those settings}, with a correlation of magnitude .742. This
relationship is highly significant (p<.01, **). Despite the a relatively low
Chi-Square (p=.115), we cannot rule out the null hypothesis.

\subsection{ISP Tracking Concerns and VPN Usage}
We exclude from our analyses one respondent who did not know what a VPN was.

Strong relationship between \emph{how an individual feels when using a VPN and
whether they use one}, with a gamma value of .786. Relatively significant
relationship (p=.169) given sample size.  Relatively significant Chi-Square
(p=.107), despite not ruling out null hypothesis conclusively.

Perfect correlation between \emph{how an individual feels while setting up a
VPN, and whether they use one}. Relatively significant relationship (p=.169)
given sample size. Relatively significant Chi-Square (p=.211), despite not
ruling out null hypothesis conclusively.

With a gamma value of .973, there eists a near perfect relationship between
\emph{how an individual feels after setting up a VPN and how they feel when
their device is using a VPN}. This relationship is highly significant
(p<.001,***), and the chi-square test of significance rules out the null
hypothesis (p<.001,***).

A perfect inverse relationship exists between \emph{how somebody feels while
setting up a VPN and how they feel if their device is not using one}. It is
relatively significant (p=.139) given the small sample size.  Relatively
significant Chi-Square (p=.183), despite not ruling out null hypothesis
conclusively.

\subsection{Password Security}
There exists a strong relationship such that \emph{if an individual used the
same passwords for most of their accounts, they feel worse about it if they use
some password protective measures}. The gamma value of this relationship is
.477, with p<.05 (*). Despite the strength of the relationship, the Chi-Square
p-value of .131 indicates that the null hypothesis cannot be ruled out.

There is a strong positive correlation betwen \emph{how an individual feels when
they reuse an old password, and how they feel if most of their accounts used the
same password}. The gamma correlation value of .733 is significant (p<.001,***).
The null hypothesis can be ruled out (Chi-Square p<.01,**).
\section{Discussion}
\subsection{Ad Tracking}
\textbf{TODO}
The inability to rule out the null hypothesis is likely due to the small sample
size of 34 respondents who were asked these questions.


%% The next two lines define the bibliography style to be used, and
%% the bibliography file.
\bibliographystyle{ACM-Reference-Format}
\bibliography{biblio}

%% If your work has an appendix, this is the place to put it.
\appendix


\end{document}
\endinput
%%
%% End of file `sample-acmtog.tex'.
